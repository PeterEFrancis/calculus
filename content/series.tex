\section{Series}

Let $(a_n)$ be a sequence, and define a new sequence $(s_k)$ of ``partial sums'' by
$$s_k=\sum_{n=1}^k a_n = a_1 + \cdots + a_k.$$
We write 
$$\lim _{k \to \infty} s_{k}=\lim _{k \to \infty} \sum_{n=1}^{k} a_{n}=\sum_{n=1}^{\infty} a_{n}=\sum a_n$$
if this limit exists, and call $\sum a_n$ a \textbf{series}. We are interested in when this quantity is finite:
\begin{itemize}
    \item If $\displaystyle\lim _{k \to \infty} s_{k}$ is finite, we say that $\sum a_n$ is \textbf{convergent} (or that $\sum a_n$ converges).
    \item  When $\displaystyle\lim _{k \to \infty} s_{k}$ is infinite, we say that $\sum a_n$ is \textbf{divergent} (or that $\sum a_n$ diverges).
\end{itemize}
There are two special cases when we can actually find the value of a series:
\begin{enumerate}
\item A series of the form $\displaystyle\sum_{n=k}^\infty r^n$ is called a \textbf{geometric series} and is convergent if and only if $|r|<1$. In that case, 
$$\sum_{n=k}^\infty r^n=\frac{r^k}{1-r}.$
For example,
$$\sum_{n=1}^\infty \left(\frac{-1}{2}\right)^n=\frac{-1}{3}.$$

\item A \textbf{telescoping series} is one where all but finitely many terms cancel. To evaluate these series, write out several terms of the sum, and see which terms are cancelled. Many times partial fractions will come in handy. For example,
$$\sum_{n=1}^\infty\frac{1}{n(n+1)}
=\sum_{n=1}^\infty\left(\frac{1}{n}-\frac{1}{n+1}\right)
=\left(\frac{1}{1}-\frac{1}{2}\right)+\left(\frac{1}{2}-\frac{1}{3}\right)+\left(\frac{1}{3}-\frac{1}{4}\right)+\dots=1.$$
\mar{Cross out terms that cancel. What does $\frac{1}{4}$ cancel with?}
\end{enumerate}

