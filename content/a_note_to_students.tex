
\section*{A Note to Students}
This text is meant to serve as a condensed resource for a student taking their first or second semester of calculus. I am writing with the assumption that the reader is familiar with some trigonometry/pre-calculus, but has perhaps forgotten some of it. While statements I write are always true, they will often not be entirely rigorous or stated with full generality. Use the big margins to make your own notes.  \mar{I'll leave prompts and questions in boxes like this.}

My goal is to communicate what I believe are the key ideas in an introductory calculus class that will help the average student succeed. Here are my suggestions for doing well in calculus (that should be cyclically repeated!):

\begin{enumerate}
\item Spend time fully understanding key concepts.
\item Identify and learn about common types of problems and exercises.
\item Find some really good examples.
\item Practice a lot!
\end{enumerate}


When you begin learning calculus, it is OK not to understand everything right away. It takes time for some concepts to sink in. Doing a lot of practice will help you start developing intuition about how to tackle new problems, even if you don't fully comprehend everything you are doing. Eventually, your problem-solving skills and your abstract understanding will both be strong, but they need to grow together and build off each other. It is good to sit and actually think about something for a while without doing any writing. Keep at it and eventually you will be able to conjure pictures and animations in your head that relate ideas and succinctly encapsulate the idea of a problem or theorem.


\section*{P.S.}
You might be thinking: ``What the \&\%\$\# is calculus?!''.

Well, let me save you a trip to Wikipedia. At its core, calculus is the study of change. Its name comes from the Latin word for ``pebbles'' because the main idea of calculus is to study things that are changing in non-linear ways by breaking them up into small pieces (like pebbles) that can be thought of as linear, and then putting them back together.

\section*{P.P.S}
Ok... I lied. The pebbles actually refer to the beads on an abacus, but a professor of mine once told me the other version, and I like that better!

