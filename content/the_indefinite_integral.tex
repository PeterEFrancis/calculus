\section{The Indefinite Integral}

Since the Fundamental Theorem of Calculus tells us that we have to understand anti-derivatives in order to compute definite integrals, this chapter will focus on this task.


\subsection{Anti-differentiation}


If $F'=f$, we call $F$ an \textbf{anti-derivative} (or \textbf{indefinite integral}) of $f$ and write
$$F(x)=\int f(x)\ dx \quad\text{or}\quad F=\int f.$$ 
Unfortunately, such a function $F$ does not always exist. \mar{Check out this \href{https://xkcd.com/2117}{XKCD Comic}.}



The first thing to understand is that unlike derivatives, anti-derivatives are not unique! This is because constant terms are always killed by differentiation. For example, the derivative of $x^2$ and $x^2+1$ are both $2x$, so both are worthy anti-derivatives for $2x$. To solve this problem, we consider \textit{the} anti-derivative of $f$ to be the collection of \textit{all} functions whose derivative is $f$. These functions only differ by a constant so we write ``$+C$'' at the end of \textit{an} anti-derivative to denote \textit{the} anti-derivative. For example, $\int 2x\ dx = x^2+C$.

Similarly to differentiation, we will build up a two-part collection of tools to tackle integration problems.

\begin{enumerate}
\item rules to break functions up and reassemble

\begin{center}
\def\arraystretch{1.5}
\begin{tabular}{@{}ll@{}}
\toprule[0.4mm]
\textbf{Scalar Multiplication} & $\int a f = a \int f$. \\
\textbf{Sum} & $\int f + \int g= \int f + \int g$ \\
\textbf{Integration by Parts} & $\int f'g = fg - \int fg' $ \\
\textbf{$u$-substitution} & $\int f'(g(x))g'(x)\ dx =  f(g(x))$ \\
% \textbf{$u$-substitution} & $\int (f'\circ g)g' =  f\circ g$ \\
\bottomrule[0.4mm]
\end{tabular}
\end{center}

\item anti-derivatives of ``basic'' functions:

\end{enumerate}


\begin{center}
\def\arraystretch{1.5}
\begin{tabular}{@{}ll@{}}
\toprule[0.4mm]
\textbf{Power Rule}
 & $\int x^a \ dx = \begin{cases}
\frac{1}{a+1}x^{a+1} + C & a \neq -1\\
\ln|x| + C & a = -1
\end{cases}$ \\
\textbf{Trig Rules}  & $\int \sin(x)\ dx = -\cos(x)+C$\\
                     & $\int \cos(x)\ dx = \sin(x)+C$ \\
\textbf{Exponential Rules} & $\int a^x\ dx = \frac{1}{\ln(a)}a^x+C$ \\
\bottomrule[0.4mm]
\end{tabular}
\end{center}

Each of the integration rules are true because of a corresponding rule for derivatives. \mar{Which differentiation rules do Integration by Parts and $u$-substitution correspond to?}

% Here are some examples using these rules:
% \begin{itemize}
%     \item $\int \sin(2x)\ dx = \frac{1}{2} \int 2\sin(2x)\ dx = -\frac{1}{2}\cos(2x) + C$ by $u$-substitution. The ``inside'' function is $2x$, and while its derivative (2) doesn't appear in the integral, we can multiply by $2$ and $\frac{1}{2}$.
% \end{itemize}


\subsection{Partial Fraction Decomposition}

When attempting to integrate a rational function, partial fraction decomposition is a common trick that comes in handy. It effectively breaks a rational function into a sum of rational functions with lower degree denominators.



\begin{center}
\def\arraystretch{1.5}
\begin{tabular}{@{}ll@{}}
\toprule[0.4mm]
      Factor in denominator & Term in decomposition\\
      \hline
      $ax+b$          & $\frac{A}{ax+b}$ \\
      $(ax+b)^k$      & $\frac{A_1}{ax+b}+\frac{A_2}{(ax+b)^2}+\dots+\frac{A_k}{(ax+b)^k}$ \\
      $ax^2+bx+c$     & $\frac{Ax+B}{ax^2+bx+c}$\\
      $(ax^2+bx+c)^k$ & $\frac{A_1x+B_1}{ax^2+bx+c}+\frac{A_2x+B_2}{(ax^2+bx+c)^2}+\dots+\frac{A_kx+B_k}{(ax^2+bx+c)^k}$ \\
\bottomrule[0.4mm]
    \end{tabular}
\end{center}


While this table is not exhaustive, it covers the range of complexity required in a calculus class.

\begin{strat}[Partial Fraction Decomposition]
    
\begin{enumerate}
  \item Factor the denominator of the original rational expression.
  \item Set the original expression equal to the sum of appropriate terms (see table).
  \item Write the sum as a single rational expression by finding a common denominator.
  \item Use the two numerators to solve a system of equations for the unknown values, $A$, $B$, $C$, etc.
\end{enumerate}
\end{strat}



\subsection{Trigonometric Substitution}

Another trick that is useful when integrating is the 



