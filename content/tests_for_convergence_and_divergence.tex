\section{Tests for Convergence and Divergence}

The following theorems are all used to determine when a series is convergent.

\begin{thm}[Divergence Test] If $\displaystyle \lim_{n\to\infty}a_n\neq 0$, then $\sum a_n$ will diverge.
\end{thm}
Be careful: the converse of this theorem is not true! (If the limit of $a_n$ is 0, that does not mean that $\sum a_n$ converges). \mar{While reading forward, look for a counter-example for the converse.} Some examples:

\begin{itemize}
\item $\lim\frac{2n}{n+1}=2\neq 0$, so $\displaystyle\sum_{n=1}^\infty \frac{2n}{n+1}$ diverges.
\item $\lim n = \infty\neq 0$, so $\displaystyle\sum_{n=1}^\infty n$ diverges.
\item $\lim\ (-1)^n$ DNE, so $\displaystyle\sum_{n=1}^\infty (-1)^n$ diverges.

\end{itemize}


\begin{thm}[Integral Test]
Suppose that $f(x)$ is a continuous, positive, and decreasing function on the interval $[k,\infty)$ and that $f(n)=a_n$. Then
$$\int_k^\infty f(x)\ dx\text{ is convergent} \iff \sum_{n=k}^\infty a_n\text{ is convergent}.$$
\end{thm}

\begin{itemize}
\item $\displaystyle \sum_{n=2}^\infty \frac{1}{n\ln(n)}$ diverges since $\displaystyle\int_2^\infty \frac{1}{x\ln(x)}\ dx$ diverges.
\item $\displaystyle \sum_{n=7}^\infty \frac{1}{n\ln(n)^2}$ converges since $\displaystyle\int_7^\infty \frac{1}{x\ln(x)^2}\ dx=\frac{1}{\ln(7)}<\infty$.
\end{itemize}


\begin{thm}[The $p$-series Test]
\phantom{}

If $k>0$, then $\displaystyle\sum_{n=k}^\infty\frac{1}{n^p}$ converges if $p>1$ and diverges if $p\leq 1$.
\end{thm}


\begin{itemize}
\item $\displaystyle \sum_{n=1}^\infty \frac{1}{n^3}$ converges since $p=3>1$.
\item $\displaystyle \sum_{n=1}^\infty \frac{1}{n}$ diverges since $p=1$.
\item $\displaystyle \sum_{n=1}^\infty \frac{1}{\sqrt{n}}$ diverges since $p=\frac{1}{2}<1$.
\end{itemize}


\begin{thm}[Comparison Test]
If $0\leq a_n\leq b_n$ for all $n$, then
$$\sum b_n\text{ converges}\implies \sum a_n\text{ converges}$$
and (by contraposition)
$$\sum a_n\text{ diverges}\implies \sum b_n\text{ diverges}.$$
\end{thm}


\begin{itemize}
\item $\displaystyle \sum_{n=1}^\infty \frac{\sin(n)^2}{n^3} \leq \sum_{n=1}^\infty \frac{1}{n^3}$ converges.
\item $\displaystyle \sum_{n=1}^\infty \frac{n^2+10}{n^3} \geq \sum_{n=1}^\infty \frac{n^2}{n^3}=\sum_{n=1}^\infty \frac{1}{n}$ diverges.
\end{itemize}



\begin{thm}[Limit Comparison Test]
Suppose that we have two series $\sum a_n$ and $\sum b_n$ with $a_n\geq 0$ and $b_n > 0$ for all $n$. Define
$$c=\lim_{n\to\infty}\frac{a_n}{b_n}.$$
If $c$ is positive and finite, then either both series converge or both series diverge.
\end{thm}



\begin{thm}[Alternating Series Test]
Suppose that we have a series $\sum a_{n}$ and either $$a_{n}=(-1)^{n} b_{n}\quad\text{ or }\quad  a_{n}=(-1)^{n+1} b_{n}$$ where $b_{n} \geq 0$ for all $n .$ Then if,
\begin{enumerate}
    \item $\displaystyle\lim _{n \to \infty} b_{n}=0$ and
    \item $\left\{b_{n}\right\}$ is a decreasing sequence
\end{enumerate}
the series $\sum a_{n}$ is convergent.
\end{thm}



\begin{thm}[Absolute Convergence Test]
$$\sum |a_n| \text{ converges}\implies \sum a_n\text{ converges}.$$
\end{thm}

If $\sum |a_n|$ converges, then $\sum a_n$ is called \textbf{absolutely convergent}. If not, $\sum a_n$ is called \textbf{conditionally convergent}.




\begin{thm}[Ratio Test]
Suppose we have the series $\sum a_{n}$. Define,
$$L=\lim _{n \to \infty} \left|\frac{a_{n+1}}{a_{n}}\right|.$$
Then,
\begin{enumerate}[leftmargin=1em]
    \item if $L<1$ the series is absolutely convergent (and hence convergent).
    \item if $L>1$ the series is divergent.
    \item  if $L=1$ the series may be divergent, conditionally convergent, or absolutely convergent.
\end{enumerate}
\end{thm}

\begin{thm}[Root Test]
Suppose that we have the series $\sum a_{n}$. Define,
$$L=\lim _{n \to \infty} \sqrt[n]{\left|a_{n}\right|}=\lim _{n \rightarrow \infty}\left|a_{n}\right|^{\frac{1}{n}}.$$
Then,
\begin{enumerate}[leftmargin=1em]
    \item if $L<1$ the series is absolutely convergent (and hence convergent).
    \item if $L>1$ the series is divergent.
    \item if $L=1$ the series may be divergent, conditionally convergent, or absolutely convergent.
\end{enumerate}
\end{thm}
