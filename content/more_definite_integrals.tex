\section{More Definite Integrals}

\subsection{Improper Integrals}

\textbf{Improper integrals} are integrals involving infinity. There are two types:
\begin{enumerate}
    \item $\pm\infty$ is in a bound of the integral, or
    \item the function being integrated has a vertical asymptote on the interval of integration
\end{enumerate}


To evaluate an improper integral, rewrite the integral as a limit of a proper integral, compute the integral as normal, and then take the limit. When evaluating integrals of the second type, make sure you use a sided limit from the inside of the bound of integration.

If the value of the improper integral is a finite quantity, we say the integral \textbf{converges}. If the value is an infinite quantity or the limit DNE, then we say the integral is \textbf{divergent}. Sometimes you might be asked the nature of the divergence (i.e. is the limit equal to $\pm\infty$ or non-existent).

Here's an example of each type:

\begin{itemize}%[leftmargin=1em]
    \item $\displaystyle \int_1^\infty \frac{1}{x^2}\ dx
     = \lim_{a\to \infty}\int_1^a \frac{1}{x^2}\ dx 
     = \lim_{a\to \infty}\left[\frac{-1}{x}\right]_1^a 
     = \lim_{a\to \infty} \left(1-\frac{1}{a}\right)
    =1.$
    \item $\displaystyle\int_{0}^{1}\frac{1}{x^{2}}\ dx 
     = \lim_{a\to 0^+} \int_{a}^{1}\frac{1}{x^{2}}\ dx 
     = \lim_{a\to 0^+}\left[\frac{-1}{x}\right]_a^1 
     = \lim_{a\to 0^+}\left(\frac{1}{a}-1\right)
     = \infty.$
\end{itemize}




\subsection{Even and Odd Functions}


