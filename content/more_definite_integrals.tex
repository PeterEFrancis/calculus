\section{More Definite Integrals}

\subsection{Improper Integrals}

\textbf{Improper integrals} are integrals involving infinity. There are two types:
\begin{enumerate}
    \item $\pm\infty$ is in a bound of the integral, or
    \item the function being integrated has a vertical asymptote on the interval of integration.
\end{enumerate}


To evaluate an improper integral, rewrite the integral as a limit of a proper integral, compute the integral as normal, and then take the limit. When evaluating integrals of the second type, make sure you use a sided limit from the inside of the bound of integration.

If the value of the improper integral is a finite quantity, we say the integral \textbf{converges}. If the value is an infinite quantity or the limit DNE, then we say the integral is \textbf{divergent}. Sometimes you might be asked the nature of the divergence (i.e. is the limit equal to $\pm\infty$ or non-existent).

Here's an example of each type:

\begin{itemize}%[leftmargin=1em]
    \item $\displaystyle \int_1^\infty \frac{1}{x^2}\ dx
     = \lim_{a\to \infty}\int_1^a \frac{1}{x^2}\ dx 
     = \lim_{a\to \infty}\left[\frac{-1}{x}\right]_1^a 
     = \lim_{a\to \infty} \left(1-\frac{1}{a}\right)
    =1.$
    \item $\displaystyle\int_{0}^{1}\frac{1}{x^{2}}\ dx 
     = \lim_{a\to 0^+} \int_{a}^{1}\frac{1}{x^{2}}\ dx 
     = \lim_{a\to 0^+}\left[\frac{-1}{x}\right]_a^1 
     = \lim_{a\to 0^+}\left(\frac{1}{a}-1\right)
     = \infty.$
\end{itemize}

\mar{For what values of $p$ does $$\int_0^1 x^p \ dx$$ converge?}

When the convergence of an integral is all you care about, the following theorem can be helpful.

\begin{theorem}
If $0\leq f(x) \leq g(x)$, then $\int_a^b f(x)\ dx \leq \int_a^b g(x)\ dx$.
\end{theorem}

In particular, this means that if $\int_a^b f(x)\ dx$ diverges, then $\int_a^b g(x)\ dx$ diverges, and if $\int_a^b g(x)\ dx$ converges, then $\int_a^b f(x)\ dx$ converges.


To finish the section, here is an algebraic property of integrals that can come in handy:
\begin{theorem} The following are always true if each integral exists and is finite.
\begin{enumerate}
\item $\displaystyle\int_a^bf(x)\ dx + \int_b^cf(x)\ dx = \int_a^c f(x)\ dx$.
\item $\displaystyle\int_a^bf(x)\ dx = -\int_b^a f(x)\ dx$.
\end{enumerate}
\end{theorem}

\mar{Use 1 to prove 2.}

\subsection{Even and Odd Functions}

A function $f$ is called \textbf{even} if $f(-x)=f(x)$, and is called \textbf{odd} if $f(-x)=-f(x)$. Even functions are symmetric about the $y$-axis, and odd functions are rotationally symmetric around the origin. Therefore, if $f$ is odd,
$$\int_{-a}^af(x)\ dx = 0$$
since the areas bounded by $f$ to the left and right of $x=0$ differ only in sign. Similarly, if $f$ is even,
$$\int_{-a}^af(x)\ dx = 2\int_{0}^af(x)\ dx$$
since the areas bounded by $f$ to the left and right of $x=0$ are the same.
Here are some examples:
\begin{itemize}
\item $\int_{-1}^1\sin(x)\ dx = 0$ since $\sin(-x)=-\sin(x)$.
\item $\int_{-1}^1\cos(x)\ dx = 2\int_0^1\cos(x)\ dx$ since $\cos(-x)=\cos(x)$.
\end{itemize}




