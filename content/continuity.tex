\section{Continuity}

\subsection{Definitions}

A function $f$ is \textbf{continuous at $a$} if three conditions are satisfied:
\begin{enumerate}
\item[(a)] $f$ is defined at $a$ (i.e. $f(a)$ makes sense)
\item[(b)] $\displaystyle\lim_{x\to a^+} f(x)= f(a)$
\item[(c)] $\displaystyle\lim_{x\to a^-} f(x)= f(a)$
\end{enumerate}

If $U$ is a subset of $\R$ and $f$ is continuous at every point in $U$, then we say that $f$ is \textbf{continuous on $U$}.

In the definition above, if any of (a), (b), or (c) are not true (or one of the sided limits doesn't exist), then $f$ is \textbf{discontinuous at $a$}. Here are three types of discontinuities that you may encounter: if $f$ is discontinuous at $a$, then
\begin{enumerate}
\item $f$ has a \textbf{removable discontinuity} at $a$ if $\displaystyle\lim _{x \to a} f(x)$ exists and is a real number.
\item $f$ has a \textbf{jump discontinuity} at $a$ if $\displaystyle\lim _{x \to a^{-}} f(x)$ and $\displaystyle\lim _{x \to a^{+}} f(x)$ both exist and are real numbers, but are different.
\item $f$ has an \textbf{infinite discontinuity} at $a$ if $\displaystyle\lim _{x \to a^{-}} f(x)=\pm \infty$ or $\displaystyle\lim _{x \to a^{+}} f(x)=\pm \infty$.
\end{enumerate}
\mar{Draw a picture of each kind of discontinuity.}

\subsection{Using Continuity}

The following types of functions are continuous at every point in their domains:
\begin{itemize}
\item polynomials
\item rational functions
\item trig and inverse trig functions
\item exponential functions
\item logarithms
\end{itemize}

It is a fact (easily verifiable from the definition of continuity and the limit laws) that the sum, product, and composition of continuous functions is continuous. Therefore, if you want to take a limit of any continuous function $f$ at a point $a$ in its domain, the limit is equal to $f(a)$ by definition of continuity.

\vspace{1em}

\noindent Here's another limit law, now that you know about continuous functions:

\begin{thm}[Composite Function Theorem]
If $f(x)$ is continuous at $L$ and $\displaystyle\lim _{x \to a} g(x)=L$, then
$$\lim _{x \to a} f(g(x))=f\left(\lim _{x \to a} g(x)\right)=f(L).$$
\end{thm}

\noindent To finish off the section, a very useful theorem:

\begin{thm}[The Intermediate Value Theorem]
Let $f$ be continuous over a closed, bounded interval $[a, b]$. If $z$ is any real number between $f(a)$ and $f(b)$, then there is a number $c$ in $[a, b]$ satisfying $f(c)=z$.
\end{thm}

\begin{figure}[h!]
\centering
\fbox{
\tikzset{every picture/.style={line width=0.75pt}}
\begin{tikzpicture}[x=0.75pt,y=0.75pt,yscale=-1,xscale=1]
%uncomment if require: \path (0,374); %set diagram left start at 0, and has height of 374

%Straight Lines [id:da38498324007722995]
\draw    (159,287) -- (517,287.2) ;
%Straight Lines [id:da5792876353049934]
\draw    (180,98.33) -- (180,328.2) ;
%Curve Lines [id:da7286746857142028]
\draw    (180,287) .. controls (366,94) and (463,235.33) .. (496,98.33) ;
%Straight Lines [id:da23865898502362382]
\draw  [dash pattern={on 0.84pt off 2.51pt}]  (254,223.67) -- (254,288.2) ;
%Straight Lines [id:da15690973094077898]
\draw  [dash pattern={on 0.84pt off 2.51pt}]  (429,167.67) -- (429,287) ;
%Straight Lines [id:da0889210997343266]
\draw  [dash pattern={on 0.84pt off 2.51pt}]  (429,167.67) -- (179,167.67) ;
%Straight Lines [id:da7783842106309002]
\draw  [dash pattern={on 0.84pt off 2.51pt}]  (254,223.67) -- (181,223.67) ;
%Straight Lines [id:da6580337268340255]
\draw  [dash pattern={on 0.84pt off 2.51pt}]  (303,197.1) -- (180,197.1) ;
%Straight Lines [id:da012629362538391309]
\draw  [dash pattern={on 0.84pt off 2.51pt}]  (303,286.33) -- (303,197.1) ;

% Text Node
\draw (248,289.4) node [anchor=north west][inner sep=0.75pt]    {$a$};
% Text Node
\draw (424,291.4) node [anchor=north west][inner sep=0.75pt]    {$b$};
% Text Node
\draw (140,216.4) node [anchor=north west][inner sep=0.75pt]    {$f( a)$};
% Text Node
\draw (139,156.4) node [anchor=north west][inner sep=0.75pt]    {$f( b)$};
% Text Node
\draw (163,186.4) node [anchor=north west][inner sep=0.75pt]    {$z$};
% Text Node
\draw (297,289.4) node [anchor=north west][inner sep=0.75pt]    {$c$};


\end{tikzpicture}
}
\caption{A Cartoon of the Intermediate Value Theorem}
\end{figure}

Phrased properly, the Intermediate Value Theorem is very intuitive. If you imagine the $x$-axis is time and the $y$-axis is position, the theorem says if you start at location $f(a)$ and end at location $f(b)$, then for any location $z$ between $f(a)$ and $f(b)$, there must have been some time that you were at $z$ (as long as you can't teleport!).

