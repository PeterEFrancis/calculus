\section{Series}

\subsection{Introduction}

Let $(a_n)$ be a sequence, and define a new sequence $(s_k)$ of ``partial sums'' by
$$s_k=\sum_{n=1}^k a_n = a_1 + \cdots + a_k.$$
We write 
$$\lim _{k \to \infty} s_{k}=\lim _{k \to \infty} \sum_{n=1}^{k} a_{n}=\sum_{n=1}^{\infty} a_{n}=\sum a_n$$
if this limit exists, and call $\sum a_n$ a \textbf{series}. We are interested in when this quantity is finite:
\begin{itemize}
    \item If $\displaystyle\lim _{k \to \infty} s_{k}$ is finite, we say that $\sum a_n$ is \textbf{convergent} (or that $\sum a_n$ converges).
    \item  When $\displaystyle\lim _{k \to \infty} s_{k}$ is infinite, we say that $\sum a_n$ is \textbf{divergent} (or that $\sum a_n$ diverges).
\end{itemize}
There are two special cases when we can actually find the value of a series:
\begin{enumerate}
\item A series of the form $\displaystyle\sum_{n=k}^\infty r^n$ is called a \textbf{geometric series} and is convergent if and only if $|r|<1$. In that case, 
$$\sum_{n=k}^\infty r^n=\frac{r^k}{1-r}.$
For example,
$$\sum_{n=1}^\infty \left(\frac{-1}{2}\right)^n=\frac{-1}{3}.$$

\item A \textbf{telescoping series} is one where all but finitely many terms cancel. To evaluate these series, write out several terms of the sum, and see which terms are cancelled. Many times partial fractions will come in handy. For example,
$$\sum_{n=1}^\infty\frac{1}{n(n+1)}
=\sum_{n=1}^\infty\left(\frac{1}{n}-\frac{1}{n+1}\right)
=\left(\frac{1}{1}-\frac{1}{2}\right)+\left(\frac{1}{2}-\frac{1}{3}\right)+\left(\frac{1}{3}-\frac{1}{4}\right)+\dots=1.$$
\mar{Cross out terms that cancel. What does $\frac{1}{4}$ cancel with?}
\end{enumerate}



\subsection{Tests for Convergence and Divergence}

The following theorems are all used to determine when a series is convergent. Note: Nothing in this section will tell you the \textit{actual value} of a series.


\begin{thm}[Divergence Test] If $\displaystyle \lim_{n\to\infty}a_n\neq 0$, then $\sum a_n$ will diverge.
\end{thm}
Be careful: the converse of this theorem is not true! (If the limit of $a_n$ is 0, that does not mean that $\sum a_n$ converges). \mar{While reading forward, look for a counter-example for the converse.} Some examples:

\begin{itemize}
\item $\lim\frac{2n}{n+1}=2\neq 0$, so $\displaystyle\sum_{n=1}^\infty \frac{2n}{n+1}$ diverges.
\item $\lim n = \infty\neq 0$, so $\displaystyle\sum_{n=1}^\infty n$ diverges.
\item $\lim\ (-1)^n$ DNE, so $\displaystyle\sum_{n=1}^\infty (-1)^n$ diverges.

\end{itemize}


\begin{thm}[Integral Test]
Suppose that $f(x)$ is a continuous, positive, and decreasing function on the interval $[k,\infty)$ and that $f(n)=a_n$. Then
$$\int_k^\infty f(x)\ dx\text{ is convergent} \iff \sum_{n=k}^\infty a_n\text{ is convergent}.$$
\end{thm}

\begin{itemize}
\item $\displaystyle \sum_{n=2}^\infty \frac{1}{n\ln(n)}$ diverges since $\displaystyle\int_2^\infty \frac{1}{x\ln(x)}\ dx$ diverges.
\item $\displaystyle \sum_{n=7}^\infty \frac{1}{n\ln(n)^2}$ converges since $\displaystyle\int_7^\infty \frac{1}{x\ln(x)^2}\ dx=\frac{1}{\ln(7)}<\infty$.
\end{itemize}


\begin{thm}[The $p$-series Test]
\phantom{}

If $k>0$, then $\displaystyle\sum_{n=k}^\infty\frac{1}{n^p}$ converges if $p>1$ and diverges if $p\leq 1$.
\end{thm}


\begin{itemize}
\item $\displaystyle \sum_{n=1}^\infty \frac{1}{n^3}$ converges since $p=3>1$.
\item $\displaystyle \sum_{n=1}^\infty \frac{1}{n}$ diverges since $p=1$.
\item $\displaystyle \sum_{n=1}^\infty \frac{1}{\sqrt{n}}$ diverges since $p=\frac{1}{2}<1$.
\end{itemize}


\begin{thm}[Comparison Test]
If $0\leq a_n\leq b_n$ for all $n$, then
$$\sum b_n\text{ converges}\implies \sum a_n\text{ converges}$$
and (by contraposition)
$$\sum a_n\text{ diverges}\implies \sum b_n\text{ diverges}.$$
\end{thm}


\begin{itemize}
\item $\displaystyle \sum_{n=1}^\infty \frac{\sin(n)^2}{n^3} \leq \sum_{n=1}^\infty \frac{1}{n^3}$ converges.
\item $\displaystyle \sum_{n=1}^\infty \frac{n^2+10}{n^3} \geq \sum_{n=1}^\infty \frac{n^2}{n^3}=\sum_{n=1}^\infty \frac{1}{n}$ diverges.
\end{itemize}



\begin{thm}[Limit Comparison Test]
Suppose that we have two series $\sum a_n$ and $\sum b_n$ with $a_n\geq 0$ and $b_n > 0$ for all $n$. Define
$$c=\lim_{n\to\infty}\frac{a_n}{b_n}.$$
If $c$ is positive and finite, then either both series converge or both series diverge.
\end{thm}


\begin{itemize}
\item $\displaystyle \sum_{n=1}^\infty \frac{1}{\sqrt{n+1}}$ diverges since $\lim \frac{1/\sqrt{n}}{1/\sqrt{n+1}}=1$ and $\displaystyle\sum_{n=1}^\infty \frac{1}{\sqrt{n}}$ diverges.
\item $\displaystyle \sum_{n=1}^\infty \frac{1}{(n+1)^2}$ converges since $\lim \frac{1/n^2}{1/(n+1)^2}=1$ and $\displaystyle\sum_{n=1}^\infty \frac{1}{n^2}$ converges.
\end{itemize}


\begin{thm}[Alternating Series Test]
Suppose that we have a series $\sum a_{n}$ and either $$a_{n}=(-1)^{n} b_{n}\quad\text{ or }\quad  a_{n}=(-1)^{n+1} b_{n}$$ where $b_{n} \geq 0$ for all $n .$ Then if,
\begin{enumerate}
    \item $\displaystyle\lim _{n \to \infty} b_{n}=0$ and
    \item $\left\{b_{n}\right\}$ is a decreasing sequence
\end{enumerate}
the series $\sum a_{n}$ is convergent.
\end{thm}


\begin{itemize}
\item $\displaystyle \sum_{n=1}^\infty \frac{(-1)^n}{n}$ converges since $\lim \frac1n=0$ and $f(x)=\frac{1}{x}$ is a decreasing function on $[1, \infty)$ (since $f'(x)=\frac{-1}{x^2}<0$ for all $x>0$) .
\end{itemize}



\begin{thm}[Absolute Convergence Test]
$$\sum |a_n| \text{ converges}\implies \sum a_n\text{ converges}.$$
\end{thm}

If $\sum |a_n|$ converges, then $\sum a_n$ is called \textbf{absolutely convergent}. If not, $\sum a_n$ is called \textbf{conditionally convergent}.


\begin{itemize}
\item $\displaystyle \sum_{n=1}^\infty \frac{(-1)^n}{n^2}$ converges since $\displaystyle \sum_{n=1}^\infty \frac{1}{n^2}$ converges.
\end{itemize}



\begin{thm}[Ratio Test]
Suppose we have the series $\sum a_{n}$. Define,
$$L=\lim _{n \to \infty} \left|\frac{a_{n+1}}{a_{n}}\right|.$$
Then,
\begin{enumerate}[leftmargin=1em]
    \item if $L<1$ the series is absolutely convergent (and hence convergent).
    \item if $L>1$ the series is divergent.
    \item  if $L=1$ the series may be divergent, conditionally convergent, or absolutely convergent.
\end{enumerate}
\end{thm}


\begin{itemize}
\item $\displaystyle \sum_{n=1}^\infty \frac{1}{n!}$ converges since $\displaystyle L=\lim\left|\frac{1/(n+1)!}{1/n!}\right|=\lim\left|\frac{1}{n+1}\right|=0<1$.
\item $\displaystyle \sum_{n=1}^\infty \frac{n!}{2^n}$ diverges since $\displaystyle L=\lim\left|\frac{(n+1)!/2^{n+1}}{n!/2^{n}}\right|=\lim\left|\frac{n+1}{2}\right|=\infty>1$.
\item The ratio test is inconclusive for $\displaystyle \sum_{n=1}^\infty \frac{1}{n^2}$ since $\displaystyle L=\lim\left|\frac{1/(n+1)^2}{1/n^2}\right|=1$.
\end{itemize}



\begin{thm}[Root Test]
Suppose that we have the series $\sum a_{n}$. Define,
$$L=\lim _{n \to \infty} \sqrt[n]{\left|a_{n}\right|}=\lim _{n \rightarrow \infty}\left|a_{n}\right|^{\frac{1}{n}}.$$
Then,
\begin{enumerate}[leftmargin=1em]
    \item if $L<1$ the series is absolutely convergent (and hence convergent).
    \item if $L>1$ the series is divergent.
    \item if $L=1$ the series may be divergent, conditionally convergent, or absolutely convergent.
\end{enumerate}
\end{thm}



\subsection{Estimating Sums}

%The main idea of this section is to compare the integral of a function with its left and right endpoint Riemann sums of width 1. It can be seen by graphing a positive, continuous, and decreasing function along with its 


\begin{thm}
If $f$ is a positive, continuous, and decreasing function on $[1,\infty)$ with $f(n)=a_n$, then for any $N$,
$$\left(\sum_{n=1}^N a_n+\int_{N+1}^\infty f(x)\ dx \right)
\leq \sum_{n=1}^\infty a_n \leq 
\left(\sum_{n=1}^N a_n+  \int_{N}^\infty f(x)\ dx\right).$$
\end{thm}

In other words, if you estimate that the value of the entire sum $\sum_{n=1}^\infty a_n$ by $\sum_{n=1}^N a_n$, you would be off by a number between $\int_{N+1}^\infty f(x)\ dx$ and $\int_{N}^\infty f(x)\ dx$. In \textit{other} other words,
$$\int_{N+1}^\infty f(x)\ dx \leq \sum_{n=N+1}^\infty a_n \leq  \int_{N}^\infty f(x)\ dx.$$

\begin{thm}

Suppose $S=\sum_{n=1}^\infty b_n (-1)^n$ where $b_n\geq 0$, and $b_n$ decreases to 0. Then for any $N$,
$$\left|S-S_N\right|\leq b_{N+1}$$
where $S_N=\sum_{n=1}^N b_n (-1)^n$.
\end{thm}
In other words, the sum of the first $N$ terms in an alternating series is different from the entire sum by at most the $N+1$st term. This means that for any odd $N$,
$$\sum_{n=1}^N b_n (-1)^n\leq\sum_{n=1}^\infty b_n (-1)^n\leq\sum_{n=1}^{N+1} b_n (-1)^n.$$







