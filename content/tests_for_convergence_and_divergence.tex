\section{Sequences}

A \textbf{sequence} of real numbers is an ordered and infinite list of real numbers.



\section{Tests for Convergence and Divergence}



Let $(a_n)$ be a sequence, and define a new sequence $(s_n)$ of ``partial sums'' by
$$s_n=\sum_{i=1}^n a_i = a_1 + \cdots + a_n.$$
We write 
$$\lim _{n \to \infty} s_{n}=\lim _{n \to \infty} \sum_{i=1}^{n} a_{i}=\sum_{i=1}^{\infty} a_{i}=\sum a_n$$
and call $\sum a_n$ an \textbf{infinite series}. We are interested in when this quantity is finite:
\begin{itemize}
    \item If $\lim _{n \to \infty} s_{n}$ is finite, we say that $\sum a_n$ is \textbf{convergent} (or that $\sum a_n$ converges).
    \item  When $\lim _{n \to \infty} s_{n}$ is infinite, we say that $\sum a_n$ is \textbf{divergent} (or that $\sum a_n$ diverges).
\end{itemize}

\vspace{.5em}

\begin{thm}[Divergence Test] If $\displaystyle \lim_{n\to\infty}a_n\neq 0$, then $\sum a_n$ will diverge.
\end{thm}


\begin{thm}[Integral Test]
Suppose that $f(x)$ is a continuous, positive, and decreasing function on the interval $[k,\infty)$ and that $f(n)=a_n$. Then
$$\int_k^\infty f(x)\ dx\text{ is convergent} \iff \sum_{n=k}^\infty a_n\text{ is convergent}.$$
\end{thm}


\begin{thm}[The $p$-series Test]
If $k>0$, then $\sum_{n=k}^\infty\frac{1}{n^p}$ converges if $p>1$ and diverges if $p\leq 1$.
\end{thm}




\begin{thm}[Comparison Test]
Suppose that we have two series $\sum a_n$ and $\sum b_n$, with $0\leq a_n\leq b_n$ for all $n$. Then
$$\sum b_n\text{ converges}\implies \sum a_n\text{ converges}$$
and (by contraposition)
$$\sum a_n\text{ diverges}\implies \sum b_n\text{ diverges}.$$
\end{thm}




\begin{thm}[Limit Comparison Test]
Suppose that we have two series $\sum a_n$ and $\sum b_n$ with $a_n\geq 0$ and $b_n > 0$ for all $n$. Define
$$c=\lim_{n\to\infty}\frac{a_n}{b_n}.$$
If $c$ is positive and finite, then either both series converge of both series diverge.
\end{thm}



\begin{thm}[Alternating Series Test]
Suppose that we have a series $\sum a_{n}$ and either $$a_{n}=(-1)^{n} b_{n}\quad\text{ or }\quad  a_{n}=(-1)^{n+1} b_{n}$$ where $b_{n} \geq 0$ for all $n .$ Then if,
\begin{enumerate}
    \item $\displaystyle\lim _{n \to \infty} b_{n}=0$ and
    \item $\left\{b_{n}\right\}$ is a decreasing sequence
\end{enumerate}
the series $\sum a_{n}$ is convergent.
\end{thm}


\begin{thm}[Absolute Convergence Test]
\phantom{}
\begin{itemize}[leftmargin=1em]
    \item If the series $\sum |a_n|$ is convergent, then $\sum a_n$ is called \textbf{absolutely convergent}, and must also be convergent.
    \item If $\sum a_n$ converges but $\sum |a_n|$ diverges, then the series $\sum a_n$ is called \textbf{conditionally convergent}.
\end{itemize}
\end{thm}


\begin{thm}[Ratio Test]
Suppose we have the series $\sum a_{n}$. Define,
$$L=\lim _{n \to \infty} \left|\frac{a_{n+1}}{a_{n}}\right|.$$
Then,
\begin{enumerate}[leftmargin=1em]
    \item if $L<1$ the series is absolutely convergent (and hence convergent).
    \item if $L>1$ the series is divergent.
    \item  if $L=1$ the series may be divergent, conditionally convergent, or absolutely convergent.
\end{enumerate}
\end{thm}

\begin{thm}[Root Test]
Suppose that we have the series $\sum a_{n}$. Define,
$$L=\lim _{n \to \infty} \sqrt[n]{\left|a_{n}\right|}=\lim _{n \rightarrow \infty}\left|a_{n}\right|^{\frac{1}{n}}.$$
Then,
\begin{enumerate}[leftmargin=1em]
    \item if $L<1$ the series is absolutely convergent (and hence convergent).
    \item if $L>1$ the series is divergent.
    \item if $L=1$ the series may be divergent, conditionally convergent, or absolutely convergent.
\end{enumerate}
\end{thm}
